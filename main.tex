\documentclass[a4paper,12pt]{article}
\usepackage[brazil]{babel}
\usepackage[utf8]{inputenc}
\usepackage{graphicx}
\usepackage{amsmath}
\usepackage{enumitem}
\usepackage{geometry}
\geometry{margin=2.5cm}

\title{Projeto de Banco de Dados - Sistema Hospitalar}
\author{Por Diogo, João Oliveira, Lucas Daniel, Lucas Gabriel, Miguel Rocha \\ e Pedro Guedes}
\date{}

\begin{document}

\maketitle

\section*{Projeto Conceitual (Minimundo)}

O Hospital Santa Saúde necessita de um sistema de banco de dados para a gestão eficiente de suas operações. Esse sistema deve armazenar informações referentes a pacientes, funcionários (médicos, enfermeiros e atendentes), especialidades médicas, consultas, exames, internações, leitos, setores, medicamentos, prescrições e agendamentos.

Cada paciente possui um número de prontuário único, nome completo, CPF, data de nascimento, telefone, e-mail e endereço. Um paciente pode realizar diversas consultas, exames e internações ao longo do tempo.

Os médicos são identificados por seu número de CRM e possuem nome, CPF, telefone, e-mail, salário e uma ou mais especialidades médicas, como cardiologia, neurologia, ortopedia, entre outras. Médicos podem realizar consultas e solicitar exames.

Os enfermeiros e atendentes também compõem o quadro de funcionários. Todos os funcionários do hospital compartilham atributos em comum, como nome, CPF, telefone, e-mail, cargo e salário. Diante disso, é possível realizar uma generalização denominada Funcionário, da qual derivam as especializações Médico, Enfermeiro e Atendente.

As consultas possuem data, horário, paciente atendido e o médico responsável. Uma consulta pode ou não gerar solicitações de exames.

Os exames possuem um código identificador, nome, tipo (imagem, sangue, urina, etc.), data de realização, paciente examinado, médico solicitante e o funcionário responsável pela execução. Um paciente pode realizar vários exames, e um exame pode ser solicitado por um único médico.

As internações são registradas com um código único, data de entrada, data de alta (quando houver), diagnóstico e o leito onde o paciente foi alocado. Cada internação está associada a um único paciente. Um leito pode receber diferentes pacientes ao longo do tempo, mas somente um por vez.

Os leitos estão localizados em setores específicos do hospital, como UTI, emergência, pediatria, entre outros. Cada setor possui um código identificador, nome e localização física no hospital, e pode conter diversos leitos.

O hospital administra uma variedade de medicamentos, cada um com código, nome, fabricante, dosagem e forma de administração. Os medicamentos podem ser prescritos tanto em consultas quanto durante internações.

As prescrições são registros que associam pacientes a medicamentos receitados, contendo informações sobre a dosagem, frequência, data da prescrição e o funcionário responsável (geralmente médico ou enfermeiro). Um paciente pode receber diversos medicamentos e um mesmo medicamento pode ser prescrito para vários pacientes.

Além disso, o hospital realiza o controle dos agendamentos, que representam compromissos futuros dos pacientes com a instituição, podendo envolver consultas, exames ou internações. Cada agendamento possui uma data, horário, tipo (consulta, exame ou internação), status (agendado, realizado ou cancelado), e pode conter observações relevantes. O agendamento está sempre vinculado a um paciente, podendo também envolver médicos, leitos e funcionários responsáveis pelo registro. Esse controle de agendamentos visa organizar a agenda da instituição, garantindo a disponibilidade de profissionais e recursos hospitalares, otimizando o atendimento e evitando conflitos de horários.

Este sistema visa garantir a rastreabilidade das ações médicas, a organização das internações, a correta administração de medicamentos e o controle de recursos hospitalares, promovendo segurança e eficiência no atendimento ao paciente.

\end{document}